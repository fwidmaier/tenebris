\documentclass[a4paper, 11pt]{article}

\usepackage{mlmodern}
\usepackage[T1]{fontenc}
\usepackage{hyperref}
\usepackage[margin=2cm]{geometry}

\usepackage{sectsty}
\allsectionsfont{\sffamily\mdseries\upshape\bfseries}

\usepackage{qrcode}
\usepackage{collectbox}


\usepackage{amsmath}
\usepackage{amsfonts}
\usepackage{amssymb}
\usepackage{amsthm}

\newcommand{\tenebris}{\texttt{tenebris}}
\newcommand{\R}{\mathbb R}
\newcommand{\Duals}[1]{#1[\varepsilon]/\left\langle\varepsilon^2\right\rangle}

\makeatletter
\newcommand{\boxedcode}{%
    \collectbox{%
        \setlength{\fboxsep}{5pt}%
        \fbox{\BOXCONTENT}%
    }%
}
\makeatother

\title{\bfseries\sffamily Some notes on \tenebris{}}
\author{Felix Widmaier}
\date{\today}

\begin{document}
    \maketitle
    \begin{abstract}
        A small and simple \texttt{Python} library that supports automatic differentiation using
        dual numbers and solving some simple equations.
    \end{abstract}

    \section*{Getting started}
\addcontentsline{toc}{section}{Getting started}
The code of \tenebris{} was originally published on
\begin{center}
    \qrcode{https://github.com/fwidmaier/tenebris}\\
    \vspace*{0.5cm}
    \href{https://github.com/fwidmaier/tenebris}{\url{https://github.com/fwidmaier/tenebris}}
\end{center}
To install the library, simply use
\begin{center}
    \boxedcode{\texttt{pip install dist/tenebris-1.0-py2.py3-none-any.whl}}
\end{center}
To build the wheel file of the library, use
\begin{center}
    \boxedcode{\texttt{python setup.py bdist\_wheel --universal}}
\end{center}
\paragraph{Dependencies.} The required packages to work with \tenebris{} are listed in \texttt{requirements.txt}.
    \textcolor{red}{Currently, there are no dependencies. Only standard libraries are used.}
\end{document}